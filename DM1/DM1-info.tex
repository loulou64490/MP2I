\documentclass[11pt,a4paper]{article}

% --- Packages de base ---
\usepackage[french]{babel}
\usepackage[T1]{fontenc}
\usepackage[utf8]{inputenc}
\usepackage{lmodern}
\usepackage{geometry}
\usepackage{setspace}
\usepackage{microtype}
\usepackage{enumitem}
\usepackage{multicol}
\usepackage{amsmath,amssymb,amsthm}
\usepackage{minted} 
\usepackage{changepage}

% --- Réglages de mise en page ---
\geometry{margin=2cm}
\setlength{\parskip}{0em}
\setlength{\parindent}{0pt}
\pagestyle{empty} 


% Macro pour des sous-puces sur plusieurs colonnes
\newenvironment{subpuces}[1][1]
{
	\begin{multicols}{#1}
		\begin{enumerate}[label=\alph*), leftmargin=*]
		}
		{
		\end{enumerate}
	\end{multicols}
}

\begin{document}

\begin{center}
	\vspace*{1em}
	{\Large\bfseries Devoir Maison n°1}\\[0.5em]
	{Louis Chauvet-Villaret -- MP2I}\\[0.25em]
	{Jeudi 6 novembre}
\end{center}

\bigskip

\begin{enumerate}[label=\textbf{Q\arabic*.}, leftmargin=*, itemsep=1em]
	\item \begin{subpuces}[4]
		      \item $110_2=6$
		      \item $100000_2=32$
		      \item $100110_2=38$
		      \item $1111_2=15$
	      \end{subpuces}

	\item \begin{subpuces}
		      \item $1\underset{n\ \text{fois}}{\underbrace{0…0}}=2^n$
		      \item $\underset{n\ \text{fois}}{\underbrace{1…1}}=2^n-1$
	      \end{subpuces}

	\item \begin{subpuces}
		      \item $1100001_2$ et $E=\{0,5,6\}$
		      \item $11111101000_2$ et $E=\{3,5,6,7,8,9,10\}$
	      \end{subpuces}

	\item $d(n)=\left\{i:\left\lfloor \frac{n}{2^i} \right\rfloor \equiv 1 \pmod{2},\ i\in\mathbb{N}\right\}$

	\item On peut représenter 256 entiers de 0 à 255.

	\item \begin{subpuces}[4]
		      \item $00000000_2=00_{16}$
		      \item $00000110_2=06_{16}$
		      \item $00100000_2=20_{16}$
		      \item $00100110_2=26_{16}$
	      \end{subpuces}

	\item \textbf{à Q12.} \
	      \begin{minted}[frame=single,linenos,fontsize=\small]{c}
uchar union(uchar n, uchar m) { return n & m; } // union est un mot clé de C...
uchar intersection(uchar n, uchar m) { return n & m; }
uchar complementaire(uchar n) { return ~n; }
bool appartient(uchar n, int i) { return n & 1 << i; }
uchar ajouter(uchar n, int i) { return n | 1 << i; }
uchar retirer(uchar n, int i) { return n ^ 1 << i; }
int cardinal(uchar n) {
	int c = 0;
	for (int i = 0; i < 8; i++) c += appartient(n, i);
	return c;}
		\end{minted}

	      \setcounter{enumi}{12}
	\item Le cardinal de $E$ est égal au nombre de bit 1 dans l'écriture binaire de $c(E)$.
	\item \begin{tabular}[t]{|l|l|l|l|l|l|} \hline
		      $n$ & $n$ (base 2) & $n-1$ & $n-1$ (base 2) & $n\&(n-1)$ & $n\&(n-1)$ (base 2) \\ \hline
		      27  & 0001 1011    & 26    & 0001 1010      & 26         & 0001 1010           \\ \hline
		      160 & 1010 0000    & 159   & 1001 1111      & 128        & 1000 0000           \\ \hline
		      32  & 0010 0000    & 31    & 0001 1111      & 0          & 0000 0000           \\ \hline
		      16  & 0001 0000    & 15    & 0000 1111      & 0          & 0000 0000           \\ \hline
		      47  & 0010 1111    & 46    & 0010 1110      & 46         & 0010 1110           \\ \hline
		      46  & 0010 1110    & 45    & 0010 1101      & 44         & 0010 1100           \\ \hline
	      \end{tabular}
	\item $d(n\&(n-1)) = d(n) \cap d(n-1)$
	\item
	      \begin{adjustwidth}{2em}{0pt}
		      \begin{minted}[frame=single,linenos,fontsize=\small]{c}
int cardinal(uchar n) {
	if (n == 0) return 0;
	return 1 + cardinal(intersection(n, n - 1));}
	\end{minted}
	      \end{adjustwidth}
	\item Dans l'implémentation naïve on effectue toujours 32 opérations (8 tours de boucle et 4 opérations par tour en comptant \texttt{i++}), alors que dans la deuxième on effectue au plus 24 opérations (3 opérations par \texttt{return} et au maximum 8 return si le cardinal vaut 8). Cette dernière est donc plus efficace lorsque l'on compare le nombre d'opération.
\end{enumerate}

\end{document}
