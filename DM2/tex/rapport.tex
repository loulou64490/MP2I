\documentclass[11pt,a4paper]{article}

% --- Packages de base ---
\usepackage[french]{babel}
\usepackage[T1]{fontenc}
\usepackage[utf8]{inputenc}
\usepackage{lmodern}
\usepackage{geometry}
\usepackage{setspace}
\usepackage{microtype}
\usepackage{enumitem}
\usepackage{multicol}
\usepackage{amsmath,amssymb,amsthm}
\usepackage{minted} 
\usepackage{changepage}

% --- Réglages de mise en page ---
\geometry{margin=2.5cm}
\setlength{\parskip}{0em}
\setlength{\parindent}{0pt}
\pagestyle{empty} 

\begin{document}

\bigskip\bigskip

\begin{center}
	{\Large\bfseries Devoir Maison n°2}\\[1em]
	{Louis Chauvet-Villaret -- MP2I}\\[0.5em]
	{Samedi 3 janvier}
\end{center}

\bigskip\bigskip

\begin{enumerate}[label=\textbf{Q\arabic*.}, leftmargin=*, itemsep=1em]
	\item $ (U_i)_{i \in \mathbb{N}} = A\sin\left(\frac{2\pi fi}{f_{ech}}\right) $
	      \setcounter{enumi}{2}
	\item \begin{minted}[fontsize=\small]{text}
000000 52 49 46 46 2e 00 00 00 57 41 56 45 66 6d 74 20
000010 10 00 00 00 01 00 01 00 22 56 00 00 44 ac 00 00
000020 02 00 10 00 64 61 74 61 0a 00 00 00 d2 03 5e 06
000030 ff ff a2 f9 ff 7f
		\end{minted}
	      \setcounter{enumi}{17}
	\item En notant $l_m=\max\{l_1,...l_n\}$, j'ai 2 boucles de $n$ tours et une
	boucle qui s'apparente à $l_m \times n$ tours donc ma fonction est de complexité $O\left(n(l_m+2)\right)$.


\end{enumerate}

\end{document}
